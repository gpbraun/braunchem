\documentclass[braun, twocolumn]{braun}
            \braunsetup{DIV=calc}
            \title{Título}
            \affiliation{Colégio e Curso Pensi, Turma IME-ITA}
            \author{Gabriel Braun}
            \logo{pensi}
            \begin{document}
            \maketitle[botrule=false]
            
            \begin{problem}
                Quando gás metano é aquecido com enxofre, são formados dissulfeto de
carbono e sulfeto de hidrogênio como únicos produtos. Uma amostra de
\(\pu{120 g}\) de metano é aquecida com \(\pu{120 g}\) enxofre.

\begin{itemize}
\item
  \textbf{Assinale} a alternativa que mais se aproxima da massa de
  dissulfeto de carbono formada.
\item
  \textbf{Determine} a alternativa que mais se aproxima da massa de
  dissulfeto de carbono formada.
\end{itemize}
            \end{problem}
        

            \begin{problem}
                Quando gás metano é aquecido com enxofre, são formados dissulfeto de
carbono e sulfeto de hidrogênio como únicos produtos. Uma amostra de
\(\pu{120 g}\) de metano é aquecida com \(\pu{120 g}\) enxofre.

\textbf{Assinale} a alternativa que mais se aproxima da massa de
dissulfeto de carbono formada.
            \end{problem}
        
            \end{document}