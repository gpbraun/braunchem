Uma mistura (\(1:15\)) de metano e ar atmosférico, a \(\pu{298 K}\) e
\(\pu{1 atm}\), entra em combustão em um reservatório adiabático,
consumindo completamente o metano. O processo ocorre sob pressão
constante e os produtos formados permanecem em fase gasosa. Considere os
data: termodinâmicos a seguir.

\begin{longtable}[]{@{}
  >{\raggedright\arraybackslash}p{(\columnwidth - 4\tabcolsep) * \real{0.33}}
  >{\raggedleft\arraybackslash}p{(\columnwidth - 4\tabcolsep) * \real{0.33}}
  >{\raggedleft\arraybackslash}p{(\columnwidth - 4\tabcolsep) * \real{0.33}}@{}}
\toprule
& \(\ce{H0}\)(\(\pu{1700 K}\)) --- \(\ce{H0}\)(\(\pu{298 K}\)) &
\(\ce{H0}\)(\(\pu{2000 K}\)) --- \(\ce{H0}\)(\(\pu{298 K}\)) \\
\midrule
\endhead
\(\ce{CO2}\) & \(\pu{17,58 kcal.mol-1}\) & \(\pu{21,90 kcal/mol-1}\) \\
\(\ce{H2O}\) & \(\pu{13,74 kcal.mol-1}\) & \(\pu{17,26 kcal.mol-1}\) \\
\(\ce{N2}\) & \(\pu{10,86 kcal.mol-1}\) & \(\pu{13,42 kcal.mol-1}\) \\
\(\ce{O2}\) & \(\pu{11,42 kcal.mol-1}\) & \(\pu{14,15 kcal.mol-1}\) \\
\bottomrule
\end{longtable}

\begin{itemize}
\tightlist
\item
  \textbf{Determine} a temperatura final do sistema.
\item
  \textbf{Determine} a concentração final de vapor d'água.
\end{itemize}

\begin{quote}
\begin{itemize}
\tightlist
\item
  \(\pu{1733 K}\)
\item
  \(\pu{5,2 mmol.L-1}\)
\end{itemize}
\end{quote}
