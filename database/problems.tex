\makeatletter\newcommand\data[1][all]{\ifnum\pdfstrcmp{#1}{all}=0\def\data@out{\{"1A02": \{"id": "1A02", "path": "database/1A/1A02", "statement": \{"html": "<p>Quando g\textbackslash{}u00e1s metano \textbackslash{}u00e9 aquecido com enxofre, s\textbackslash{}u00e3o formados dissulfeto de carbono e sulfeto de hidrog\textbackslash{}u00eanio como \textbackslash{}u00fanicos produtos. Uma amostra de \$\textbackslash{}\textbackslash{}pu\{120 g\}\$ de metano \textbackslash{}u00e9 aquecida com \$\textbackslash{}\textbackslash{}pu\{120 g\}\$ enxofre.</p>\textbackslash{}n<ul>\textbackslash{}n<li><strong>Assinale</strong> a alternativa que mais se aproxima da massa de dissulfeto de carbono formada.</li>\textbackslash{}n<li><strong>Determine</strong> a alternativa que mais se aproxima da massa de dissulfeto de carbono formada.</li>\textbackslash{}n</ul>\textbackslash{}n\textbackslash{}n", "md": "Quando g\textbackslash{}u00e1s metano \textbackslash{}u00e9 aquecido com enxofre, s\textbackslash{}u00e3o formados dissulfeto de\textbackslash{}ncarbono e sulfeto de hidrog\textbackslash{}u00eanio como \textbackslash{}u00fanicos produtos. Uma amostra de\textbackslash{}n\$\textbackslash{}\textbackslash{}pu\{120 g\}\$ de metano \textbackslash{}u00e9 aquecida com \$\textbackslash{}\textbackslash{}pu\{120 g\}\$ enxofre.\textbackslash{}n\textbackslash{}n{-}   **Assinale** a alternativa que mais se aproxima da massa de\textbackslash{}n    dissulfeto de carbono formada.\textbackslash{}n{-}   **Determine** a alternativa que mais se aproxima da massa de\textbackslash{}n    dissulfeto de carbono formada.", "tex": "Quando g\textbackslash{}u00e1s metano \textbackslash{}u00e9 aquecido com enxofre, s\textbackslash{}u00e3o formados dissulfeto de\textbackslash{}ncarbono e sulfeto de hidrog\textbackslash{}u00eanio como \textbackslash{}u00fanicos produtos. Uma amostra de\textbackslash{}n\textbackslash{}\textbackslash{}(\textbackslash{}\textbackslash{}pu\{120 g\}\textbackslash{}\textbackslash{}) de metano \textbackslash{}u00e9 aquecida com \textbackslash{}\textbackslash{}(\textbackslash{}\textbackslash{}pu\{120 g\}\textbackslash{}\textbackslash{}) enxofre.\textbackslash{}n\textbackslash{}n\textbackslash{}\textbackslash{}begin\{itemize\}\textbackslash{}n\textbackslash{}\textbackslash{}tightlist\textbackslash{}n\textbackslash{}\textbackslash{}item\textbackslash{}n  \textbackslash{}\textbackslash{}textbf\{Assinale\} a alternativa que mais se aproxima da massa de\textbackslash{}n  dissulfeto de carbono formada.\textbackslash{}n\textbackslash{}\textbackslash{}item\textbackslash{}n  \textbackslash{}\textbackslash{}textbf\{Determine\} a alternativa que mais se aproxima da massa de\textbackslash{}n  dissulfeto de carbono formada.\textbackslash{}n\textbackslash{}\textbackslash{}end\{itemize\}"\}, "solution": \{"html": "\textbackslash{}n\textbackslash{}n{-}10\textbackslash{}n11\textbackslash{}n12\textbackslash{}n\textbackslash{}nAssim, substituindo a em \$\$ \textbackslash{}\textbackslash{}sum f(x) \textbackslash{}\textbackslash{}mathrm\{d\} x = \textbackslash{}\textbackslash{}int \textbackslash{}\textbackslash{}Delta x \$\$ Temos o seguinte: A fun\textbackslash{}u00e7\textbackslash{}u00e3o fica \$f(x) = 3\$\textbackslash{}n", "md": "{-}10 11 12 Assim, substituindo a em\textbackslash{}n\$\$ \textbackslash{}\textbackslash{}sum f(x) \textbackslash{}\textbackslash{}mathrm\{d\} x = \textbackslash{}\textbackslash{}int \textbackslash{}\textbackslash{}Delta x \$\$ Temos o seguinte: A fun\textbackslash{}u00e7\textbackslash{}u00e3o\textbackslash{}nfica \$f(x) = 3\$", "tex": "{-}10 11 12 Assim, substituindo a em\textbackslash{}n\textbackslash{}\textbackslash{}{[} \textbackslash{}\textbackslash{}sum f(x) \textbackslash{}\textbackslash{}mathrm\{d\} x = \textbackslash{}\textbackslash{}int \textbackslash{}\textbackslash{}Delta x \textbackslash{}\textbackslash{}{]} Temos o seguinte: A fun\textbackslash{}u00e7\textbackslash{}u00e3o\textbackslash{}nfica \textbackslash{}\textbackslash{}(f(x) = 3\textbackslash{}\textbackslash{})"\}, "obj": {-}1, "answer": 10, "options": {[}{]}, "data": {[}"Hf{-}CO2(g)"{]}\}, "1A01": \{"id": "1A01", "path": "database/1A/1A01", "statement": \{"html": "<p>Quando g\textbackslash{}u00e1s metano \textbackslash{}u00e9 aquecido com enxofre, s\textbackslash{}u00e3o formados dissulfeto de carbono e sulfeto de hidrog\textbackslash{}u00eanio como \textbackslash{}u00fanicos produtos. Uma amostra de \$\textbackslash{}\textbackslash{}pu\{120 g\}\$ de metano \textbackslash{}u00e9 aquecida com \$\textbackslash{}\textbackslash{}pu\{120 g\}\$ enxofre.</p>\textbackslash{}n<p><strong>Assinale</strong> a alternativa que mais se aproxima da massa de dissulfeto de carbono formada.</p>\textbackslash{}n\textbackslash{}n\textbackslash{}n", "md": "Quando g\textbackslash{}u00e1s metano \textbackslash{}u00e9 aquecido com enxofre, s\textbackslash{}u00e3o formados dissulfeto de\textbackslash{}ncarbono e sulfeto de hidrog\textbackslash{}u00eanio como \textbackslash{}u00fanicos produtos. Uma amostra de\textbackslash{}n\$\textbackslash{}\textbackslash{}pu\{120 g\}\$ de metano \textbackslash{}u00e9 aquecida com \$\textbackslash{}\textbackslash{}pu\{120 g\}\$ enxofre.\textbackslash{}n\textbackslash{}n**Assinale** a alternativa que mais se aproxima da massa de dissulfeto\textbackslash{}nde carbono formada.", "tex": "Quando g\textbackslash{}u00e1s metano \textbackslash{}u00e9 aquecido com enxofre, s\textbackslash{}u00e3o formados dissulfeto de\textbackslash{}ncarbono e sulfeto de hidrog\textbackslash{}u00eanio como \textbackslash{}u00fanicos produtos. Uma amostra de\textbackslash{}n\textbackslash{}\textbackslash{}(\textbackslash{}\textbackslash{}pu\{120 g\}\textbackslash{}\textbackslash{}) de metano \textbackslash{}u00e9 aquecida com \textbackslash{}\textbackslash{}(\textbackslash{}\textbackslash{}pu\{120 g\}\textbackslash{}\textbackslash{}) enxofre.\textbackslash{}n\textbackslash{}n\textbackslash{}\textbackslash{}textbf\{Assinale\} a alternativa que mais se aproxima da massa de\textbackslash{}ndissulfeto de carbono formada."\}, "solution": \{"html": "\textbackslash{}n\textbackslash{}n{-}10\textbackslash{}n11\textbackslash{}n12\textbackslash{}n\textbackslash{}nAssim, substituindo a em \$\$ \textbackslash{}\textbackslash{}sum f(x) \textbackslash{}\textbackslash{}mathrm\{d\} x = \textbackslash{}\textbackslash{}int \textbackslash{}\textbackslash{}Delta x \$\$ Temos o seguinte: A fun\textbackslash{}u00e7\textbackslash{}u00e3o fica \$f(x) = 3\$\textbackslash{}n", "md": "{-}10 11 12 Assim, substituindo a em\textbackslash{}n\$\$ \textbackslash{}\textbackslash{}sum f(x) \textbackslash{}\textbackslash{}mathrm\{d\} x = \textbackslash{}\textbackslash{}int \textbackslash{}\textbackslash{}Delta x \$\$ Temos o seguinte: A fun\textbackslash{}u00e7\textbackslash{}u00e3o\textbackslash{}nfica \$f(x) = 3\$", "tex": "{-}10 11 12 Assim, substituindo a em\textbackslash{}n\textbackslash{}\textbackslash{}{[} \textbackslash{}\textbackslash{}sum f(x) \textbackslash{}\textbackslash{}mathrm\{d\} x = \textbackslash{}\textbackslash{}int \textbackslash{}\textbackslash{}Delta x \textbackslash{}\textbackslash{}{]} Temos o seguinte: A fun\textbackslash{}u00e7\textbackslash{}u00e3o\textbackslash{}nfica \textbackslash{}\textbackslash{}(f(x) = 3\textbackslash{}\textbackslash{})"\}, "obj": \{"html": "1", "md": "1", "tex": "1"\}, "answer": \{"html": "<li>\textbackslash{}n71</li>", "md": "71", "tex": "71"\}, "options": {[}\{"html": "<li>\textbackslash{}n35</li>", "md": "35", "tex": "35"\}, \{"html": "<li>\textbackslash{}n71</li>", "md": "71", "tex": "71"\}, \{"html": "<li>\textbackslash{}n142</li>", "md": "142", "tex": "142"\}, \{"html": "<li>\textbackslash{}n285</li>", "md": "285", "tex": "285"\}, \{"html": "<li>\textbackslash{}n570</li>", "md": "570", "tex": "570"\}{]}, "data": {[}"Hf{-}CO2(g)"{]}\}\}}\else\ifnum\pdfstrcmp{#1}{1A02}=0\let\data@out\data@I\else\ifnum\pdfstrcmp{#1}{1A01}=0\let\data@out\data@II\else\def\data@out{??}\fi\fi\fi\data@out}\newcommand\data@I[1][all]{\ifnum\pdfstrcmp{#1}{all}=0\def\data@I@out{\{"id": "1A02", "path": "database/1A/1A02", "statement": \{"html": "<p>Quando g\textbackslash{}u00e1s metano \textbackslash{}u00e9 aquecido com enxofre, s\textbackslash{}u00e3o formados dissulfeto de carbono e sulfeto de hidrog\textbackslash{}u00eanio como \textbackslash{}u00fanicos produtos. Uma amostra de \$\textbackslash{}\textbackslash{}pu\{120 g\}\$ de metano \textbackslash{}u00e9 aquecida com \$\textbackslash{}\textbackslash{}pu\{120 g\}\$ enxofre.</p>\textbackslash{}n<ul>\textbackslash{}n<li><strong>Assinale</strong> a alternativa que mais se aproxima da massa de dissulfeto de carbono formada.</li>\textbackslash{}n<li><strong>Determine</strong> a alternativa que mais se aproxima da massa de dissulfeto de carbono formada.</li>\textbackslash{}n</ul>\textbackslash{}n\textbackslash{}n", "md": "Quando g\textbackslash{}u00e1s metano \textbackslash{}u00e9 aquecido com enxofre, s\textbackslash{}u00e3o formados dissulfeto de\textbackslash{}ncarbono e sulfeto de hidrog\textbackslash{}u00eanio como \textbackslash{}u00fanicos produtos. Uma amostra de\textbackslash{}n\$\textbackslash{}\textbackslash{}pu\{120 g\}\$ de metano \textbackslash{}u00e9 aquecida com \$\textbackslash{}\textbackslash{}pu\{120 g\}\$ enxofre.\textbackslash{}n\textbackslash{}n{-}   **Assinale** a alternativa que mais se aproxima da massa de\textbackslash{}n    dissulfeto de carbono formada.\textbackslash{}n{-}   **Determine** a alternativa que mais se aproxima da massa de\textbackslash{}n    dissulfeto de carbono formada.", "tex": "Quando g\textbackslash{}u00e1s metano \textbackslash{}u00e9 aquecido com enxofre, s\textbackslash{}u00e3o formados dissulfeto de\textbackslash{}ncarbono e sulfeto de hidrog\textbackslash{}u00eanio como \textbackslash{}u00fanicos produtos. Uma amostra de\textbackslash{}n\textbackslash{}\textbackslash{}(\textbackslash{}\textbackslash{}pu\{120 g\}\textbackslash{}\textbackslash{}) de metano \textbackslash{}u00e9 aquecida com \textbackslash{}\textbackslash{}(\textbackslash{}\textbackslash{}pu\{120 g\}\textbackslash{}\textbackslash{}) enxofre.\textbackslash{}n\textbackslash{}n\textbackslash{}\textbackslash{}begin\{itemize\}\textbackslash{}n\textbackslash{}\textbackslash{}tightlist\textbackslash{}n\textbackslash{}\textbackslash{}item\textbackslash{}n  \textbackslash{}\textbackslash{}textbf\{Assinale\} a alternativa que mais se aproxima da massa de\textbackslash{}n  dissulfeto de carbono formada.\textbackslash{}n\textbackslash{}\textbackslash{}item\textbackslash{}n  \textbackslash{}\textbackslash{}textbf\{Determine\} a alternativa que mais se aproxima da massa de\textbackslash{}n  dissulfeto de carbono formada.\textbackslash{}n\textbackslash{}\textbackslash{}end\{itemize\}"\}, "solution": \{"html": "\textbackslash{}n\textbackslash{}n{-}10\textbackslash{}n11\textbackslash{}n12\textbackslash{}n\textbackslash{}nAssim, substituindo a em \$\$ \textbackslash{}\textbackslash{}sum f(x) \textbackslash{}\textbackslash{}mathrm\{d\} x = \textbackslash{}\textbackslash{}int \textbackslash{}\textbackslash{}Delta x \$\$ Temos o seguinte: A fun\textbackslash{}u00e7\textbackslash{}u00e3o fica \$f(x) = 3\$\textbackslash{}n", "md": "{-}10 11 12 Assim, substituindo a em\textbackslash{}n\$\$ \textbackslash{}\textbackslash{}sum f(x) \textbackslash{}\textbackslash{}mathrm\{d\} x = \textbackslash{}\textbackslash{}int \textbackslash{}\textbackslash{}Delta x \$\$ Temos o seguinte: A fun\textbackslash{}u00e7\textbackslash{}u00e3o\textbackslash{}nfica \$f(x) = 3\$", "tex": "{-}10 11 12 Assim, substituindo a em\textbackslash{}n\textbackslash{}\textbackslash{}{[} \textbackslash{}\textbackslash{}sum f(x) \textbackslash{}\textbackslash{}mathrm\{d\} x = \textbackslash{}\textbackslash{}int \textbackslash{}\textbackslash{}Delta x \textbackslash{}\textbackslash{}{]} Temos o seguinte: A fun\textbackslash{}u00e7\textbackslash{}u00e3o\textbackslash{}nfica \textbackslash{}\textbackslash{}(f(x) = 3\textbackslash{}\textbackslash{})"\}, "obj": {-}1, "answer": 10, "options": {[}{]}, "data": {[}"Hf{-}CO2(g)"{]}\}}\else\ifnum\pdfstrcmp{#1}{id}=0\def\data@I@out{1A02}\else\ifnum\pdfstrcmp{#1}{path}=0\def\data@I@out{database/1A/1A02}\else\ifnum\pdfstrcmp{#1}{statement}=0\let\data@I@out\data@III\else\ifnum\pdfstrcmp{#1}{solution}=0\let\data@I@out\data@IV\else\ifnum\pdfstrcmp{#1}{obj}=0\def\data@I@out{{-}1}\else\ifnum\pdfstrcmp{#1}{answer}=0\def\data@I@out{10}\else\ifnum\pdfstrcmp{#1}{options}=0\let\data@I@out\data@V\else\ifnum\pdfstrcmp{#1}{data}=0\let\data@I@out\data@VI\else\def\data@I@out{??}\fi\fi\fi\fi\fi\fi\fi\fi\fi\data@I@out}\newcommand\data@II[1][all]{\ifnum\pdfstrcmp{#1}{all}=0\def\data@II@out{\{"id": "1A01", "path": "database/1A/1A01", "statement": \{"html": "<p>Quando g\textbackslash{}u00e1s metano \textbackslash{}u00e9 aquecido com enxofre, s\textbackslash{}u00e3o formados dissulfeto de carbono e sulfeto de hidrog\textbackslash{}u00eanio como \textbackslash{}u00fanicos produtos. Uma amostra de \$\textbackslash{}\textbackslash{}pu\{120 g\}\$ de metano \textbackslash{}u00e9 aquecida com \$\textbackslash{}\textbackslash{}pu\{120 g\}\$ enxofre.</p>\textbackslash{}n<p><strong>Assinale</strong> a alternativa que mais se aproxima da massa de dissulfeto de carbono formada.</p>\textbackslash{}n\textbackslash{}n\textbackslash{}n", "md": "Quando g\textbackslash{}u00e1s metano \textbackslash{}u00e9 aquecido com enxofre, s\textbackslash{}u00e3o formados dissulfeto de\textbackslash{}ncarbono e sulfeto de hidrog\textbackslash{}u00eanio como \textbackslash{}u00fanicos produtos. Uma amostra de\textbackslash{}n\$\textbackslash{}\textbackslash{}pu\{120 g\}\$ de metano \textbackslash{}u00e9 aquecida com \$\textbackslash{}\textbackslash{}pu\{120 g\}\$ enxofre.\textbackslash{}n\textbackslash{}n**Assinale** a alternativa que mais se aproxima da massa de dissulfeto\textbackslash{}nde carbono formada.", "tex": "Quando g\textbackslash{}u00e1s metano \textbackslash{}u00e9 aquecido com enxofre, s\textbackslash{}u00e3o formados dissulfeto de\textbackslash{}ncarbono e sulfeto de hidrog\textbackslash{}u00eanio como \textbackslash{}u00fanicos produtos. Uma amostra de\textbackslash{}n\textbackslash{}\textbackslash{}(\textbackslash{}\textbackslash{}pu\{120 g\}\textbackslash{}\textbackslash{}) de metano \textbackslash{}u00e9 aquecida com \textbackslash{}\textbackslash{}(\textbackslash{}\textbackslash{}pu\{120 g\}\textbackslash{}\textbackslash{}) enxofre.\textbackslash{}n\textbackslash{}n\textbackslash{}\textbackslash{}textbf\{Assinale\} a alternativa que mais se aproxima da massa de\textbackslash{}ndissulfeto de carbono formada."\}, "solution": \{"html": "\textbackslash{}n\textbackslash{}n{-}10\textbackslash{}n11\textbackslash{}n12\textbackslash{}n\textbackslash{}nAssim, substituindo a em \$\$ \textbackslash{}\textbackslash{}sum f(x) \textbackslash{}\textbackslash{}mathrm\{d\} x = \textbackslash{}\textbackslash{}int \textbackslash{}\textbackslash{}Delta x \$\$ Temos o seguinte: A fun\textbackslash{}u00e7\textbackslash{}u00e3o fica \$f(x) = 3\$\textbackslash{}n", "md": "{-}10 11 12 Assim, substituindo a em\textbackslash{}n\$\$ \textbackslash{}\textbackslash{}sum f(x) \textbackslash{}\textbackslash{}mathrm\{d\} x = \textbackslash{}\textbackslash{}int \textbackslash{}\textbackslash{}Delta x \$\$ Temos o seguinte: A fun\textbackslash{}u00e7\textbackslash{}u00e3o\textbackslash{}nfica \$f(x) = 3\$", "tex": "{-}10 11 12 Assim, substituindo a em\textbackslash{}n\textbackslash{}\textbackslash{}{[} \textbackslash{}\textbackslash{}sum f(x) \textbackslash{}\textbackslash{}mathrm\{d\} x = \textbackslash{}\textbackslash{}int \textbackslash{}\textbackslash{}Delta x \textbackslash{}\textbackslash{}{]} Temos o seguinte: A fun\textbackslash{}u00e7\textbackslash{}u00e3o\textbackslash{}nfica \textbackslash{}\textbackslash{}(f(x) = 3\textbackslash{}\textbackslash{})"\}, "obj": \{"html": "1", "md": "1", "tex": "1"\}, "answer": \{"html": "<li>\textbackslash{}n71</li>", "md": "71", "tex": "71"\}, "options": {[}\{"html": "<li>\textbackslash{}n35</li>", "md": "35", "tex": "35"\}, \{"html": "<li>\textbackslash{}n71</li>", "md": "71", "tex": "71"\}, \{"html": "<li>\textbackslash{}n142</li>", "md": "142", "tex": "142"\}, \{"html": "<li>\textbackslash{}n285</li>", "md": "285", "tex": "285"\}, \{"html": "<li>\textbackslash{}n570</li>", "md": "570", "tex": "570"\}{]}, "data": {[}"Hf{-}CO2(g)"{]}\}}\else\ifnum\pdfstrcmp{#1}{id}=0\def\data@II@out{1A01}\else\ifnum\pdfstrcmp{#1}{path}=0\def\data@II@out{database/1A/1A01}\else\ifnum\pdfstrcmp{#1}{statement}=0\let\data@II@out\data@VII\else\ifnum\pdfstrcmp{#1}{solution}=0\let\data@II@out\data@VIII\else\ifnum\pdfstrcmp{#1}{obj}=0\let\data@II@out\data@IX\else\ifnum\pdfstrcmp{#1}{answer}=0\let\data@II@out\data@X\else\ifnum\pdfstrcmp{#1}{options}=0\let\data@II@out\data@XI\else\ifnum\pdfstrcmp{#1}{data}=0\let\data@II@out\data@XII\else\def\data@II@out{??}\fi\fi\fi\fi\fi\fi\fi\fi\fi\data@II@out}\newcommand\data@III[1][all]{\ifnum\pdfstrcmp{#1}{all}=0\def\data@III@out{\{"html": "<p>Quando g\textbackslash{}u00e1s metano \textbackslash{}u00e9 aquecido com enxofre, s\textbackslash{}u00e3o formados dissulfeto de carbono e sulfeto de hidrog\textbackslash{}u00eanio como \textbackslash{}u00fanicos produtos. Uma amostra de \$\textbackslash{}\textbackslash{}pu\{120 g\}\$ de metano \textbackslash{}u00e9 aquecida com \$\textbackslash{}\textbackslash{}pu\{120 g\}\$ enxofre.</p>\textbackslash{}n<ul>\textbackslash{}n<li><strong>Assinale</strong> a alternativa que mais se aproxima da massa de dissulfeto de carbono formada.</li>\textbackslash{}n<li><strong>Determine</strong> a alternativa que mais se aproxima da massa de dissulfeto de carbono formada.</li>\textbackslash{}n</ul>\textbackslash{}n\textbackslash{}n", "md": "Quando g\textbackslash{}u00e1s metano \textbackslash{}u00e9 aquecido com enxofre, s\textbackslash{}u00e3o formados dissulfeto de\textbackslash{}ncarbono e sulfeto de hidrog\textbackslash{}u00eanio como \textbackslash{}u00fanicos produtos. Uma amostra de\textbackslash{}n\$\textbackslash{}\textbackslash{}pu\{120 g\}\$ de metano \textbackslash{}u00e9 aquecida com \$\textbackslash{}\textbackslash{}pu\{120 g\}\$ enxofre.\textbackslash{}n\textbackslash{}n{-}   **Assinale** a alternativa que mais se aproxima da massa de\textbackslash{}n    dissulfeto de carbono formada.\textbackslash{}n{-}   **Determine** a alternativa que mais se aproxima da massa de\textbackslash{}n    dissulfeto de carbono formada.", "tex": "Quando g\textbackslash{}u00e1s metano \textbackslash{}u00e9 aquecido com enxofre, s\textbackslash{}u00e3o formados dissulfeto de\textbackslash{}ncarbono e sulfeto de hidrog\textbackslash{}u00eanio como \textbackslash{}u00fanicos produtos. Uma amostra de\textbackslash{}n\textbackslash{}\textbackslash{}(\textbackslash{}\textbackslash{}pu\{120 g\}\textbackslash{}\textbackslash{}) de metano \textbackslash{}u00e9 aquecida com \textbackslash{}\textbackslash{}(\textbackslash{}\textbackslash{}pu\{120 g\}\textbackslash{}\textbackslash{}) enxofre.\textbackslash{}n\textbackslash{}n\textbackslash{}\textbackslash{}begin\{itemize\}\textbackslash{}n\textbackslash{}\textbackslash{}tightlist\textbackslash{}n\textbackslash{}\textbackslash{}item\textbackslash{}n  \textbackslash{}\textbackslash{}textbf\{Assinale\} a alternativa que mais se aproxima da massa de\textbackslash{}n  dissulfeto de carbono formada.\textbackslash{}n\textbackslash{}\textbackslash{}item\textbackslash{}n  \textbackslash{}\textbackslash{}textbf\{Determine\} a alternativa que mais se aproxima da massa de\textbackslash{}n  dissulfeto de carbono formada.\textbackslash{}n\textbackslash{}\textbackslash{}end\{itemize\}"\}}\else\ifnum\pdfstrcmp{#1}{html}=0\def\data@III@out{<p>Quando gás metano é aquecido com enxofre, são formados dissulfeto de carbono e sulfeto de hidrogênio como únicos produtos. Uma amostra de \$\textbackslash{}pu\{120 g\}\$ de metano é aquecida com \$\textbackslash{}pu\{120 g\}\$ enxofre.</p>\newline%
<ul>\newline%
<li><strong>Assinale</strong> a alternativa que mais se aproxima da massa de dissulfeto de carbono formada.</li>\newline%
<li><strong>Determine</strong> a alternativa que mais se aproxima da massa de dissulfeto de carbono formada.</li>\newline%
</ul>\newline%
\newline%
}\else\ifnum\pdfstrcmp{#1}{md}=0\def\data@III@out{Quando gás metano é aquecido com enxofre, são formados dissulfeto de\newline%
carbono e sulfeto de hidrogênio como únicos produtos. Uma amostra de\newline%
\$\textbackslash{}pu\{120 g\}\$ de metano é aquecida com \$\textbackslash{}pu\{120 g\}\$ enxofre.\newline%
\newline%
{-}   **Assinale** a alternativa que mais se aproxima da massa de\newline%
    dissulfeto de carbono formada.\newline%
{-}   **Determine** a alternativa que mais se aproxima da massa de\newline%
    dissulfeto de carbono formada.}\else\ifnum\pdfstrcmp{#1}{tex}=0\def\data@III@out{Quando gás metano é aquecido com enxofre, são formados dissulfeto de\newline%
carbono e sulfeto de hidrogênio como únicos produtos. Uma amostra de\newline%
\textbackslash{}(\textbackslash{}pu\{120 g\}\textbackslash{}) de metano é aquecida com \textbackslash{}(\textbackslash{}pu\{120 g\}\textbackslash{}) enxofre.\newline%
\newline%
\textbackslash{}begin\{itemize\}\newline%
\textbackslash{}tightlist\newline%
\textbackslash{}item\newline%
  \textbackslash{}textbf\{Assinale\} a alternativa que mais se aproxima da massa de\newline%
  dissulfeto de carbono formada.\newline%
\textbackslash{}item\newline%
  \textbackslash{}textbf\{Determine\} a alternativa que mais se aproxima da massa de\newline%
  dissulfeto de carbono formada.\newline%
\textbackslash{}end\{itemize\}}\else\def\data@III@out{??}\fi\fi\fi\fi\data@III@out}\newcommand\data@IV[1][all]{\ifnum\pdfstrcmp{#1}{all}=0\def\data@IV@out{\{"html": "\textbackslash{}n\textbackslash{}n{-}10\textbackslash{}n11\textbackslash{}n12\textbackslash{}n\textbackslash{}nAssim, substituindo a em \$\$ \textbackslash{}\textbackslash{}sum f(x) \textbackslash{}\textbackslash{}mathrm\{d\} x = \textbackslash{}\textbackslash{}int \textbackslash{}\textbackslash{}Delta x \$\$ Temos o seguinte: A fun\textbackslash{}u00e7\textbackslash{}u00e3o fica \$f(x) = 3\$\textbackslash{}n", "md": "{-}10 11 12 Assim, substituindo a em\textbackslash{}n\$\$ \textbackslash{}\textbackslash{}sum f(x) \textbackslash{}\textbackslash{}mathrm\{d\} x = \textbackslash{}\textbackslash{}int \textbackslash{}\textbackslash{}Delta x \$\$ Temos o seguinte: A fun\textbackslash{}u00e7\textbackslash{}u00e3o\textbackslash{}nfica \$f(x) = 3\$", "tex": "{-}10 11 12 Assim, substituindo a em\textbackslash{}n\textbackslash{}\textbackslash{}{[} \textbackslash{}\textbackslash{}sum f(x) \textbackslash{}\textbackslash{}mathrm\{d\} x = \textbackslash{}\textbackslash{}int \textbackslash{}\textbackslash{}Delta x \textbackslash{}\textbackslash{}{]} Temos o seguinte: A fun\textbackslash{}u00e7\textbackslash{}u00e3o\textbackslash{}nfica \textbackslash{}\textbackslash{}(f(x) = 3\textbackslash{}\textbackslash{})"\}}\else\ifnum\pdfstrcmp{#1}{html}=0\def\data@IV@out{\newline%
\newline%
{-}10\newline%
11\newline%
12\newline%
\newline%
Assim, substituindo a em \$\$ \textbackslash{}sum f(x) \textbackslash{}mathrm\{d\} x = \textbackslash{}int \textbackslash{}Delta x \$\$ Temos o seguinte: A função fica \$f(x) = 3\$\newline%
}\else\ifnum\pdfstrcmp{#1}{md}=0\def\data@IV@out{{-}10 11 12 Assim, substituindo a em\newline%
\$\$ \textbackslash{}sum f(x) \textbackslash{}mathrm\{d\} x = \textbackslash{}int \textbackslash{}Delta x \$\$ Temos o seguinte: A função\newline%
fica \$f(x) = 3\$}\else\ifnum\pdfstrcmp{#1}{tex}=0\def\data@IV@out{{-}10 11 12 Assim, substituindo a em\newline%
\textbackslash{}{[} \textbackslash{}sum f(x) \textbackslash{}mathrm\{d\} x = \textbackslash{}int \textbackslash{}Delta x \textbackslash{}{]} Temos o seguinte: A função\newline%
fica \textbackslash{}(f(x) = 3\textbackslash{})}\else\def\data@IV@out{??}\fi\fi\fi\fi\data@IV@out}\newcommand\data@V[1][all]{\ifnum\pdfstrcmp{#1}{all}=0\def\data@V@out{{[}{]}}\else\def\data@V@out{??}\fi\data@V@out}\newcommand\data@VI[1][all]{\ifnum\pdfstrcmp{#1}{all}=0\def\data@VI@out{{[}"Hf{-}CO2(g)"{]}}\else\ifnum\pdfstrcmp{#1}{0}=0\def\data@VI@out{Hf{-}CO2(g)}\else\def\data@VI@out{??}\fi\fi\data@VI@out}\newcommand\data@VII[1][all]{\ifnum\pdfstrcmp{#1}{all}=0\def\data@VII@out{\{"html": "<p>Quando g\textbackslash{}u00e1s metano \textbackslash{}u00e9 aquecido com enxofre, s\textbackslash{}u00e3o formados dissulfeto de carbono e sulfeto de hidrog\textbackslash{}u00eanio como \textbackslash{}u00fanicos produtos. Uma amostra de \$\textbackslash{}\textbackslash{}pu\{120 g\}\$ de metano \textbackslash{}u00e9 aquecida com \$\textbackslash{}\textbackslash{}pu\{120 g\}\$ enxofre.</p>\textbackslash{}n<p><strong>Assinale</strong> a alternativa que mais se aproxima da massa de dissulfeto de carbono formada.</p>\textbackslash{}n\textbackslash{}n\textbackslash{}n", "md": "Quando g\textbackslash{}u00e1s metano \textbackslash{}u00e9 aquecido com enxofre, s\textbackslash{}u00e3o formados dissulfeto de\textbackslash{}ncarbono e sulfeto de hidrog\textbackslash{}u00eanio como \textbackslash{}u00fanicos produtos. Uma amostra de\textbackslash{}n\$\textbackslash{}\textbackslash{}pu\{120 g\}\$ de metano \textbackslash{}u00e9 aquecida com \$\textbackslash{}\textbackslash{}pu\{120 g\}\$ enxofre.\textbackslash{}n\textbackslash{}n**Assinale** a alternativa que mais se aproxima da massa de dissulfeto\textbackslash{}nde carbono formada.", "tex": "Quando g\textbackslash{}u00e1s metano \textbackslash{}u00e9 aquecido com enxofre, s\textbackslash{}u00e3o formados dissulfeto de\textbackslash{}ncarbono e sulfeto de hidrog\textbackslash{}u00eanio como \textbackslash{}u00fanicos produtos. Uma amostra de\textbackslash{}n\textbackslash{}\textbackslash{}(\textbackslash{}\textbackslash{}pu\{120 g\}\textbackslash{}\textbackslash{}) de metano \textbackslash{}u00e9 aquecida com \textbackslash{}\textbackslash{}(\textbackslash{}\textbackslash{}pu\{120 g\}\textbackslash{}\textbackslash{}) enxofre.\textbackslash{}n\textbackslash{}n\textbackslash{}\textbackslash{}textbf\{Assinale\} a alternativa que mais se aproxima da massa de\textbackslash{}ndissulfeto de carbono formada."\}}\else\ifnum\pdfstrcmp{#1}{html}=0\def\data@VII@out{<p>Quando gás metano é aquecido com enxofre, são formados dissulfeto de carbono e sulfeto de hidrogênio como únicos produtos. Uma amostra de \$\textbackslash{}pu\{120 g\}\$ de metano é aquecida com \$\textbackslash{}pu\{120 g\}\$ enxofre.</p>\newline%
<p><strong>Assinale</strong> a alternativa que mais se aproxima da massa de dissulfeto de carbono formada.</p>\newline%
\newline%
\newline%
}\else\ifnum\pdfstrcmp{#1}{md}=0\def\data@VII@out{Quando gás metano é aquecido com enxofre, são formados dissulfeto de\newline%
carbono e sulfeto de hidrogênio como únicos produtos. Uma amostra de\newline%
\$\textbackslash{}pu\{120 g\}\$ de metano é aquecida com \$\textbackslash{}pu\{120 g\}\$ enxofre.\newline%
\newline%
**Assinale** a alternativa que mais se aproxima da massa de dissulfeto\newline%
de carbono formada.}\else\ifnum\pdfstrcmp{#1}{tex}=0\def\data@VII@out{Quando gás metano é aquecido com enxofre, são formados dissulfeto de\newline%
carbono e sulfeto de hidrogênio como únicos produtos. Uma amostra de\newline%
\textbackslash{}(\textbackslash{}pu\{120 g\}\textbackslash{}) de metano é aquecida com \textbackslash{}(\textbackslash{}pu\{120 g\}\textbackslash{}) enxofre.\newline%
\newline%
\textbackslash{}textbf\{Assinale\} a alternativa que mais se aproxima da massa de\newline%
dissulfeto de carbono formada.}\else\def\data@VII@out{??}\fi\fi\fi\fi\data@VII@out}\newcommand\data@VIII[1][all]{\ifnum\pdfstrcmp{#1}{all}=0\def\data@VIII@out{\{"html": "\textbackslash{}n\textbackslash{}n{-}10\textbackslash{}n11\textbackslash{}n12\textbackslash{}n\textbackslash{}nAssim, substituindo a em \$\$ \textbackslash{}\textbackslash{}sum f(x) \textbackslash{}\textbackslash{}mathrm\{d\} x = \textbackslash{}\textbackslash{}int \textbackslash{}\textbackslash{}Delta x \$\$ Temos o seguinte: A fun\textbackslash{}u00e7\textbackslash{}u00e3o fica \$f(x) = 3\$\textbackslash{}n", "md": "{-}10 11 12 Assim, substituindo a em\textbackslash{}n\$\$ \textbackslash{}\textbackslash{}sum f(x) \textbackslash{}\textbackslash{}mathrm\{d\} x = \textbackslash{}\textbackslash{}int \textbackslash{}\textbackslash{}Delta x \$\$ Temos o seguinte: A fun\textbackslash{}u00e7\textbackslash{}u00e3o\textbackslash{}nfica \$f(x) = 3\$", "tex": "{-}10 11 12 Assim, substituindo a em\textbackslash{}n\textbackslash{}\textbackslash{}{[} \textbackslash{}\textbackslash{}sum f(x) \textbackslash{}\textbackslash{}mathrm\{d\} x = \textbackslash{}\textbackslash{}int \textbackslash{}\textbackslash{}Delta x \textbackslash{}\textbackslash{}{]} Temos o seguinte: A fun\textbackslash{}u00e7\textbackslash{}u00e3o\textbackslash{}nfica \textbackslash{}\textbackslash{}(f(x) = 3\textbackslash{}\textbackslash{})"\}}\else\ifnum\pdfstrcmp{#1}{html}=0\def\data@VIII@out{\newline%
\newline%
{-}10\newline%
11\newline%
12\newline%
\newline%
Assim, substituindo a em \$\$ \textbackslash{}sum f(x) \textbackslash{}mathrm\{d\} x = \textbackslash{}int \textbackslash{}Delta x \$\$ Temos o seguinte: A função fica \$f(x) = 3\$\newline%
}\else\ifnum\pdfstrcmp{#1}{md}=0\def\data@VIII@out{{-}10 11 12 Assim, substituindo a em\newline%
\$\$ \textbackslash{}sum f(x) \textbackslash{}mathrm\{d\} x = \textbackslash{}int \textbackslash{}Delta x \$\$ Temos o seguinte: A função\newline%
fica \$f(x) = 3\$}\else\ifnum\pdfstrcmp{#1}{tex}=0\def\data@VIII@out{{-}10 11 12 Assim, substituindo a em\newline%
\textbackslash{}{[} \textbackslash{}sum f(x) \textbackslash{}mathrm\{d\} x = \textbackslash{}int \textbackslash{}Delta x \textbackslash{}{]} Temos o seguinte: A função\newline%
fica \textbackslash{}(f(x) = 3\textbackslash{})}\else\def\data@VIII@out{??}\fi\fi\fi\fi\data@VIII@out}\newcommand\data@IX[1][all]{\ifnum\pdfstrcmp{#1}{all}=0\def\data@IX@out{\{"html": "1", "md": "1", "tex": "1"\}}\else\ifnum\pdfstrcmp{#1}{html}=0\def\data@IX@out{1}\else\ifnum\pdfstrcmp{#1}{md}=0\def\data@IX@out{1}\else\ifnum\pdfstrcmp{#1}{tex}=0\def\data@IX@out{1}\else\def\data@IX@out{??}\fi\fi\fi\fi\data@IX@out}\newcommand\data@X[1][all]{\ifnum\pdfstrcmp{#1}{all}=0\def\data@X@out{\{"html": "<li>\textbackslash{}n71</li>", "md": "71", "tex": "71"\}}\else\ifnum\pdfstrcmp{#1}{html}=0\def\data@X@out{<li>\newline%
71</li>}\else\ifnum\pdfstrcmp{#1}{md}=0\def\data@X@out{71}\else\ifnum\pdfstrcmp{#1}{tex}=0\def\data@X@out{71}\else\def\data@X@out{??}\fi\fi\fi\fi\data@X@out}\newcommand\data@XI[1][all]{\ifnum\pdfstrcmp{#1}{all}=0\def\data@XI@out{{[}\{"html": "<li>\textbackslash{}n35</li>", "md": "35", "tex": "35"\}, \{"html": "<li>\textbackslash{}n71</li>", "md": "71", "tex": "71"\}, \{"html": "<li>\textbackslash{}n142</li>", "md": "142", "tex": "142"\}, \{"html": "<li>\textbackslash{}n285</li>", "md": "285", "tex": "285"\}, \{"html": "<li>\textbackslash{}n570</li>", "md": "570", "tex": "570"\}{]}}\else\ifnum\pdfstrcmp{#1}{0}=0\let\data@XI@out\data@XIII\else\ifnum\pdfstrcmp{#1}{1}=0\let\data@XI@out\data@XIV\else\ifnum\pdfstrcmp{#1}{2}=0\let\data@XI@out\data@XV\else\ifnum\pdfstrcmp{#1}{3}=0\let\data@XI@out\data@XVI\else\ifnum\pdfstrcmp{#1}{4}=0\let\data@XI@out\data@XVII\else\def\data@XI@out{??}\fi\fi\fi\fi\fi\fi\data@XI@out}\newcommand\data@XII[1][all]{\ifnum\pdfstrcmp{#1}{all}=0\def\data@XII@out{{[}"Hf{-}CO2(g)"{]}}\else\ifnum\pdfstrcmp{#1}{0}=0\def\data@XII@out{Hf{-}CO2(g)}\else\def\data@XII@out{??}\fi\fi\data@XII@out}\newcommand\data@XIII[1][all]{\ifnum\pdfstrcmp{#1}{all}=0\def\data@XIII@out{\{"html": "<li>\textbackslash{}n35</li>", "md": "35", "tex": "35"\}}\else\ifnum\pdfstrcmp{#1}{html}=0\def\data@XIII@out{<li>\newline%
35</li>}\else\ifnum\pdfstrcmp{#1}{md}=0\def\data@XIII@out{35}\else\ifnum\pdfstrcmp{#1}{tex}=0\def\data@XIII@out{35}\else\def\data@XIII@out{??}\fi\fi\fi\fi\data@XIII@out}\newcommand\data@XIV[1][all]{\ifnum\pdfstrcmp{#1}{all}=0\def\data@XIV@out{\{"html": "<li>\textbackslash{}n71</li>", "md": "71", "tex": "71"\}}\else\ifnum\pdfstrcmp{#1}{html}=0\def\data@XIV@out{<li>\newline%
71</li>}\else\ifnum\pdfstrcmp{#1}{md}=0\def\data@XIV@out{71}\else\ifnum\pdfstrcmp{#1}{tex}=0\def\data@XIV@out{71}\else\def\data@XIV@out{??}\fi\fi\fi\fi\data@XIV@out}\newcommand\data@XV[1][all]{\ifnum\pdfstrcmp{#1}{all}=0\def\data@XV@out{\{"html": "<li>\textbackslash{}n142</li>", "md": "142", "tex": "142"\}}\else\ifnum\pdfstrcmp{#1}{html}=0\def\data@XV@out{<li>\newline%
142</li>}\else\ifnum\pdfstrcmp{#1}{md}=0\def\data@XV@out{142}\else\ifnum\pdfstrcmp{#1}{tex}=0\def\data@XV@out{142}\else\def\data@XV@out{??}\fi\fi\fi\fi\data@XV@out}\newcommand\data@XVI[1][all]{\ifnum\pdfstrcmp{#1}{all}=0\def\data@XVI@out{\{"html": "<li>\textbackslash{}n285</li>", "md": "285", "tex": "285"\}}\else\ifnum\pdfstrcmp{#1}{html}=0\def\data@XVI@out{<li>\newline%
285</li>}\else\ifnum\pdfstrcmp{#1}{md}=0\def\data@XVI@out{285}\else\ifnum\pdfstrcmp{#1}{tex}=0\def\data@XVI@out{285}\else\def\data@XVI@out{??}\fi\fi\fi\fi\data@XVI@out}\newcommand\data@XVII[1][all]{\ifnum\pdfstrcmp{#1}{all}=0\def\data@XVII@out{\{"html": "<li>\textbackslash{}n570</li>", "md": "570", "tex": "570"\}}\else\ifnum\pdfstrcmp{#1}{html}=0\def\data@XVII@out{<li>\newline%
570</li>}\else\ifnum\pdfstrcmp{#1}{md}=0\def\data@XVII@out{570}\else\ifnum\pdfstrcmp{#1}{tex}=0\def\data@XVII@out{570}\else\def\data@XVII@out{??}\fi\fi\fi\fi\data@XVII@out}\makeatother