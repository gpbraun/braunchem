Em sistemas contendo um grande número de elétrons, não existe um único
estado ocupado mais energético, já que vários estados possuem a mesma
energia. O nível de Fermi corresponde aos estados ocupados que possuem
maior energia. Considere uma folha quadrada, de lado \(L = 25 nm\), de
grafeno. Os elétrons \(\pi\) desse sistema podem ser modelados como
partículas em uma caixa bidimensional, cujos níveis de energia são dados
por:

\[ E(n_1, n_2) =
\dfrac{h^{2}}{8m\_{e}L^{2}}\left(n\_{1}^{2}+n\_{2}^{2}\right) \]

Em que dois elétrons ocupam um mesmo nível de energia.

\begin{itemize}

\item
  \textbf{Determine} o número de elétrons \(\pi\) nesse sistema.
\item
  \textbf{Determine} a energia do nível de Fermi desse sistema.
\item
  \textbf{Correlacione} a condutividade e o tamanho da folha de grafeno.
\end{itemize}
