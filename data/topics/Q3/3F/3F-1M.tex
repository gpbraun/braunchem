\begin{chemscheme}
\setchemfig{+ sep left=1.2em, + sep right=1.2em}
\schemestart
    \chemfig[fixed length=true]
        {
            H-[1]@{H1}\charge{45=\:, 135=\:}{O}-[-1]@{O1}{\color{red}H}
        }
    \arrow{0}[,0]\+\arrow{0}[,0]
    \chemfig[fixed length=true]
        {
            @{O2}\charge{0:1.5pt=\:, 90:1.5pt=\:, 180:1.5pt=\:, 270:1.5pt=\:, 40:6pt=$\scriptstyle2 -$}{O}
        }
    \arrow{->}
    \chemfig[fixed length=true]
        {
            H-\charge{270:1.5pt=\:, 90:1.5pt=\:, 0:1.5pt=\:, 40:4pt=$\scriptstyle -$}{O}
        }
    \arrow{0}[,0.1]\+\arrow{0}[,0]
    \chemfig[fixed length=true]
        {
            {\color{red}H}-\charge{270:1.5pt=\:, 90:1.5pt=\:, 0:1.5pt=\:, 40:4pt=$\scriptstyle -$}{O}
        }
\schemestop
\chemmove[shorten <=3pt, shorten >=3pt]
    {
        \draw[green, semithick, shorten <=5pt] (O2) ..controls+ (135:8mm) and+ (60:8mm).. (O1);
        \draw[green, semithick] (O1) ..controls+ (-120:6mm) and+ (-90:6mm).. (H1);
    } 
\end{chemscheme}

    