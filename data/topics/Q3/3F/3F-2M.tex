\begin{chemscheme}
\setchemfig{+ sep left=1.2em, + sep right=1.2em}
\schemestart
    \chemfig[fixed length=true]
        {
            H-[1]@{N1}\charge{90:2pt=\:}{N}(-[-4]H)-[-1]H
        }
    \arrow{0}[,0]\+\arrow{0}[,0]
    \chemfig[fixed length=true]
        {
            @{H1}{\color{red}H}-[1]@{O1}\charge{45=\:, 135=\:}{O}-[-1]H
        }
    \arrow{->}
    \chemfig[fixed length=true]
        {
            H-[1]\charge{45:1pt=$\scriptstyle +$}{N}(-[3]{\color{red}H})(-[-4]H)-[-1]H
        }
    \arrow{0}[,-0.1]\+\arrow{0}[,0.1]
    \chemfig[fixed length=true]
        {
            \charge{270:1.5pt=\:, 90:1.5pt=\:, 180:1.5pt=\:, 140:4pt=$\scriptstyle -$}{O}|H
        }
\schemestop
\chemmove[shorten <=3pt, shorten >=3pt]
    {
        \draw[green, semithick, shorten <=3pt] (N1) ..controls+ (30:10mm) and+ (120:10mm).. (H1);
        \draw[green, semithick] (H1) ..controls+ (-60:6mm) and+ (-90:8mm).. (O1);
    } 
\end{chemscheme}
    