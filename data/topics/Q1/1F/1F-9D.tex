\pgfdeclarelayer{back bonds}
\pgfdeclarelayer{back atoms}
\pgfsetlayers{back atoms, back bonds, main}
\begin{chemscheme}
\begin{tikzpicture}
    \node (C1) [3Datom=gray] at (0,0) {};

    \begin{pgfonlayer}{back bonds}
    \node (L1) [3Dbond={ -30}{1.0}] at (C1) {};
    \node (L2) [3Dbond={-150}{1.0}] at (C1) {};
    \node (L3) [3Dbond={ +90}{1.0}] at (C1) {};
    \end{pgfonlayer}

    \begin{scope}[every node/.style={3Datom=red}]
        \node at ($(L1.top) + (L1.bottom)$) {};
        \node at ($(L2.top) + (L2.bottom)$) {};
        \node at ($(L3.top) + (L3.bottom)$) {};
    \end{scope}

    \draw (L3.south east) 
        to [out=0, in=60, edge node={node [above, shift={(6pt,2pt)}] {\footnotesize\ang{120}}}] 
        (L1.north east);

    \node (lewis) [anchor=east, shift={(-3em, 5pt)}] at (C1) 
        { 
            \charge{30:4pt=$2-$}{\chemleft[
                \chemfig[fixed length=true]
                    {
                        C
                            (=[3]\charge{30:1pt=\:, 150:1pt=\:}{O})
                            (-[-1]\charge{60:1pt=\:, -30:1pt=\:, -120:1pt=\:}{O})
                            (-[7]\charge{-60:1pt=\:, 210:1pt=\:, 120:1pt=\:}{O})
                    }
            \chemright]}
            \quad =
        };
\end{tikzpicture}
\end{chemscheme}
